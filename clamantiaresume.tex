%% start of file `template.tex'.
%% Copyright 2006-2013 Xavier Danaux (xdanaux@gmail.com).
%
% This work may be distributed and/or modified under the
% conditions of the LaTeX Project Public License version 1.3c,
% available at http://www.latex-project.org/lppl/.

% possible options include:
% font size ('10pt', '11pt' and '12pt')
% paper size ('a4paper', 'letterpaper', 'a5paper', 'legalpaper', 'executivepaper' and 'landscape')
% font family ('sans' and 'roman')
\documentclass[10pt,letterpaper,roman]{moderncv}

% moderncv themes
\moderncvstyle{banking}                                 % style options are 'casual' (default), 'classic', 'oldstyle' and 'banking'
\moderncvcolor{black}                                     % color options 'blue' (default), 'orange', 'green', 'red', 'purple', 'grey' and 'black'
\renewcommand{\familydefault}{\rmdefault} % to set the default font; use '\sfdefault' for the default sans serif font, '\rmdefault' for the default roman one, or any tex font name
\nopagenumbers{}

% character encoding
%\usepackage[utf8]{inputenc} % if you are not using xelatex ou lualatex, replace by the encoding you are using

% adjust the page margins
\usepackage[scale=0.80, top=1.5cm, bottom=1.5cm]{geometry}

\def\cpp{C{}\texttt{++}}

%--
%   personal data
%--
\name{Charlie}{Lamantia}
\address{Street}{City State Zip}{Country}
\phone[mobile]{Phone}
\email{email}
\homepage{webpage}

%--
%   content
%--
\begin{document}
\makecvtitle

\vspace*{-2\baselineskip}
\section{Experience}

\cventry{July 2014--Present}{Embedded Systems Engineer}{Sonomed Escalon}{Stoneham, MA}{}{}
\begin{itemize}
\item Ophthalmic ultrasound device component selection and schematic capture design
\item Developed GUI in Windows Presentation Foundation interfacing USB ultrasound board
\item Characterized legacy medical device and optimized algorithm to be emulated in C\#
\item Developed kernel level device driver for Windows Driver Kit
\item Updated firmware on existing product line to improve power consumption
\end{itemize}
\smallskip

\cventry{2012--2014}{Senior Engineer}{Electronic Communications Lab at the University of Florida}{Gainesville, FL}{}{}
\begin{itemize}
\item Development of RADAR system prototyping platform for defense research
\item Design of RADAR tracking algorithm for VHDL implementation
\item Developed a connected component labeling system with unity find data structure in VHDL
\item Developed GUI in the Qt framework with Python for data acquisition and display
\item Developed data acquisition system in VHDL and captured simulated RADAR data in lab setting
\end{itemize}
\smallskip

\cventry{2010--2012}{Research Assistant}{Electronic Communications Lab at the University of Florida}{Gainesville, FL}{}{}
\begin{itemize}
\item Designed modular DSP system with dynamic routing to facilitate rapid prototyping in VHDL
\item Implemented legacy RADAR algorithms in programmable logic devices
\item Scaled designs to fit in smaller footprint chips while maintaining performance metrics
\item Solved timing issues in inter-processor communication of legacy devices
\item Optimized existing filter designs for increased precision and operating frequency
\item Troubleshooting and maintenance of on-site hardware including PCs and test equipment
\end{itemize}
\smallskip

\cventry{2001--2007}{Technical Support}{Glatting Jackson Kercher Anglin, Inc.}{Orlando, FL}{}{Support and maintenance of workstations and printing solutions for landscape architecture firm.}
\cventry{2000--2001}{Technical Support}{Convergsys}{Lake Mary, FL}{}{Phone center support for Dell computers, consumer line.}

\section{Education}
\cventry{2010--2012}{Master of Science in Electrical Engineering}{University of Florida}{}{}{}
\cventry{2008--2010}{Bachelor of Science in Electrical Engineering}{University of Florida}{}{Cum Laude}{}

\section{Graduate Research Project}
\cvitem{Title}{\emph{A Novel Architecture for Digital RADAR Systems}}
\cvitem{Supervisors}{Lab Director James Kurtz \& Associate Lab Director Chad Overman}
\cvitem{Description}{A digital architecture was designed with enhanced modularity, in order to facilitate rapid prototyping of the RADAR signal processing chain.}

\section{Skills}
\cvitem{Programming Languages}{VHDL, C\#, \cpp, C, Python, MatLab, Basic, and \LaTeX}
\cvitem{Development Environments}{Altera Quartus, Altera Qsys, Xilinx ISE, Lattice Diamond, Model-Sim, Code Composer Studio, Visual Studio, Eclipse, Altium Designer, Eagle Pro, LTSpice, OrCAD, Multisim, Ultiboard, and Simulink}
\cvitem{Operating Systems}{Windows, Linux, Android, and iOS}
\cvitem{Hobbies}{Rock climbing, Soccer, Cycling, Music, Art, Film, Game development, Board games, and Video games}

\end{document}


%% end of file `template.tex'.
